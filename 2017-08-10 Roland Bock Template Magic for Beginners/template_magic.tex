\documentclass[aspectratio=169,xcolor=dvipsnames]{beamer}
%\usepackage{beamerthemeBerkeley}
% Use either the one above or the one below
%\usetheme{bars}
%\usetheme[height=7mm]{Rochester}
%\usetheme{PaloAlto}
\usetheme{Warsaw}
\usenavigationsymbolstemplate{} % no navigation bar
\usecolortheme[RGB={75,182,56}]{structure}

\setbeamersize{text margin left=0.5cm}
\setbeamersize{text margin right=0.5cm}
%\setbeamertemplate{blocks}[rounded]% [shadow=false]

%\usepackage{fancyvrb}
\usepackage{listings}

\definecolor{bg}{rgb}{0.95,0.95,0.95}
\definecolor{keywordcolor}{rgb}{0.31,0.53,0.23}
\definecolor{stringcolor}{RGB}{90,28,37}
\definecolor{commentcolor}{RGB}{122,73,80}

\lstset{% general command to set parameter(s)
  %language=C++,
  basicstyle=\scriptsize\ttfamily, % print whole listing small
  columns=fixed,
  basewidth=0.5em,
  tabsize=2,
  keywordstyle=\color{keywordcolor}, %\bfseries\underbar,
  identifierstyle=\color{black}, % nothing happens
  commentstyle=\color{commentcolor}, % white comments
  stringstyle=\color{stringcolor},
  showstringspaces=false,
  backgroundcolor=\color{bg},
  morecomment=[l]{//},
  moredelim=[il][\color{red}]{==|},
  moredelim=[is][\color{red}]{==>}{<==},
  moredelim=[il][\color{RoyalBlue}]{--|},
  moredelim=[is][\color{RoyalBlue}]{-->}{<--}
} % no special string spaces


\expandafter\def\expandafter\insertshorttitle\expandafter{%
  \insertshorttitle\hfill%
  \insertframenumber}

\newcommand{\comment}[1]{}

\comment{
\lstset{literate=%
   *{0}{{{\color{red!20!violet}0}}}1
    {1}{{{\color{red!20!violet}1}}}1
    {2}{{{\color{red!20!violet}2}}}1
    {3}{{{\color{red!20!violet}3}}}1
    {4}{{{\color{red!20!violet}4}}}1
    {5}{{{\color{red!20!violet}5}}}1
    {6}{{{\color{red!20!violet}6}}}1
    {7}{{{\color{red!20!violet}7}}}1
    {8}{{{\color{red!20!violet}8}}}1
    {9}{{{\color{red!20!violet}9}}}1
}
}
\lstset{literate={`}{{$^{\backprime}$}}1}

\newcommand{\navigationframe}[1]{
\begin{frame}{#1}
\tableofcontents[currentsection]
\end{frame}
}

\title{Template Magic For Beginners}
\author{Dr.\ Roland Bock}
\institute{http://ppro.com\\rbock at eudoxos dot de\\~\\https://github.com/rbock/sqlpp11}
\date{Dresden, 2017-08-10}

\begin{document}

\frame{\titlepage}

\begin{frame}[fragile]{Template Magic}
    \begin{block}{What is it?}
        \pause
        It is not magic.
    \end{block}
\end{frame}

\begin{frame}[fragile]{Template Meta Programming}
\large What is it?
\pause
Wikipedia:\\
{\sl Template metaprogramming (TMP) is a metaprogramming technique in which
  templates are used by a compiler to generate temporary source code, which
  is merged by the compiler with the rest of the source code and then compiled.\\
  \pause The output of these templates include compile-time constants, data
    structures, and complete functions.}\\
    ~\\
    \pause
    Let's take a look.
\end{frame}

\begin{frame}[fragile]{std::vector::reserve}
    \begin{block}{}
        \begin{lstlisting}
auto v = std::vector<T>{};
...
-->v.reserve(n)<--; // What's happening here?
        \end{lstlisting}
    \end{block}
\end{frame}

\begin{frame}[fragile]{std::vector::reserve}
    \begin{block}{}
        \begin{lstlisting}
auto reserve(size_type n) -> void
{
    if (capacity < n)
    {
        if (-->T is bool<--)
            do something
        \end{lstlisting}
\pause
        \begin{lstlisting}
        else if (-->T is trivially copyable<--)
            do something else
        \end{lstlisting}
\pause
        \begin{lstlisting}
        else if (-->T is noexcept movable<--)
            do yet another thing
        \end{lstlisting}
\pause
        \begin{lstlisting}
        else
            oh, come on!
    }
}
        \end{lstlisting}
        \pause
        Can't do all of this at runtime.
    \end{block}
\end{frame}

\begin{frame}[fragile]{std::vector::reserve}
    \begin{block}{We'll try to solve this with TMP}
        \begin{lstlisting}
auto reserve(size_type n) -> void
{
    if (capacity < n)
    {
        if (-->T is bool<--)
            do something

        else if (-->T is trivially copyable<--)
            do something else

        else if (-->T is noexcept movable<--)
            do yet another thing

        else
            oh, come on!
    }
}
        \end{lstlisting}
    \end{block}
    \pause First, we will reduce the problem.
\end{frame}

\begin{frame}[fragile]{Template Meta Programming}
    \begin{block}{Reduced problem}
        \begin{lstlisting}
auto reserve(size_type n) -> void
{
    if (capacity < n)
    {
        if (-->T is bool<--)
            do something
        else
            do something else
    }
}
        \end{lstlisting}
    \end{block}
\end{frame}

\begin{frame}[fragile]{Partial Specialization}
    \begin{block}{Let's do something about bool}
        \begin{lstlisting}
template<class T, class Allocator = std::allocator<T>>
class vector;
        \end{lstlisting}
\pause
        \begin{lstlisting}
template<class Allocator>
class vector<-->bool<--, Allocator>
{...};
        \end{lstlisting}
    \end{block}
\end{frame}

\begin{frame}[fragile]{Template Meta programming}
    \begin{block}{Bool is out of the way now}
        \begin{lstlisting}
auto reserve(size_type n) -> void
{
    if (capacity < n)
    {
        if (-->T is trivially copyable<--)
            do something

        else if (-->T is noexcept movable<--)
            do another thing

        else
            oh, come on!
    }
}
        \end{lstlisting}
    \end{block}
\end{frame}

\begin{frame}[fragile]{Template Meta Programming}
    \begin{block}{Factor out common stuff}
        \begin{lstlisting}
auto reserve(size_type n) -> void
{
    if (capacity < n)
    {
        auto new_memory = allocate_new_memory(n);

        if (-->T is trivially copyable<--)
            do_memcpy(new_memory);

        else if (-->T is noexcept movable<--)
            move_every_item(new_memory);

        else
            copy_every_item(new_memory);
    }
}
        \end{lstlisting}
    \end{block}
\end{frame}

\begin{frame}[fragile]{Template Meta Programming}
    \begin{block}{Make it a binary problem again}
        \begin{lstlisting}
auto reserve(size_type n) -> void
{
    if (capacity < n)
    {
        auto new_memory = allocate_new_memory(n);

        if (==>T is trivially copyable<==)
            do_memmove(new_memory);

        else
            copy_or_move_every_item(new_memory);
    }
}
        \end{lstlisting}
    \end{block}
    How can we evaluate the condition?
\end{frame}

\begin{frame}[fragile]{Type Traits}
    \begin{block}{From header $<$type\_traits$>$}
        \begin{lstlisting}
template< class T, T v >
struct integral_constant
{
    static constexpr T value = v;
};
        \end{lstlisting}
\pause
        \begin{lstlisting}
using true_type = std::integral_constant<bool, true>;
using false_type = std::integral_constant<bool, false>;
        \end{lstlisting}
    \end{block}
\end{frame}

\begin{frame}[fragile]{Type Traits}
    \begin{block}{From header $<$type\_traits$>$}
        \begin{lstlisting}
template <class T>
struct is_trivially_copyable
{
    using type = ...; // Either true_type or false_type
}
        \end{lstlisting}
\pause
        \begin{lstlisting}
template <class T>
using is_trivially_copyable_t = typename is_trivially_copyable<T>::type;
        \end{lstlisting}
    \end{block}
\end{frame}

\begin{frame}[fragile]{Tag Dispatch}
    \begin{block}{Using type traits as function call arguments}
        \begin{lstlisting}
auto reserve(size_type n) -> void
{
    if (capacity < n)
    {
        auto new_memory = allocate_new_memory(n);

        move_mem_or_elements(==>is_trivially_copyable_t<T>{}<==, begin(), end(), new_memory);
    }
}
        \end{lstlisting}
    \end{block}
\end{frame}

\begin{frame}[fragile]{Tag Dispatch}
    \begin{block}{Function overloads on tag types}
        \begin{lstlisting}
template<typename It, typename Data>
auto move_mem_or_elements(==>true_type<==, It begin, It, end, Data& new_memory) -> void
{
    // Do memmove
}
        \end{lstlisting}
\pause
        \begin{lstlisting}
template<typename It, typename Data>
auto move_mem_or_elements(==>false_type<==, It begin, It, end, Data& new_memory) -> void
{
    copy_or_move(begin, end, new_memory);
}
        \end{lstlisting}
    \end{block}
\end{frame}

\begin{frame}[fragile]{SFINAE}
    \pause
    \begin{block}{From header $<$type\_traits$>$}
        \begin{lstlisting}
template <bool B, class T = void>
struct enable_if
{
};
        \end{lstlisting}
\pause
        \begin{lstlisting}
template <class T>
struct enable_if<true, T>
{
    using type = T;
};
        \end{lstlisting}
\pause
        \begin{lstlisting}
template <bool B, class T = void>
using enable_if_t = typename enable_if<B, T>::type;
        \end{lstlisting}
    \end{block}
\end{frame}

\begin{frame}[fragile]{SFINAE}
    \begin{block}{Turning overloads on/off with enable\_if}
        \pause
        \begin{lstlisting}
template <typename Iterator, typename Data>
auto -->copy_or_move<--(Iterator begin, Iterator, end, Data& new_memory) -> void;
        \end{lstlisting}
\pause
        \begin{lstlisting}
template <typename T, typename Data,
          ==>typename = enable_if_t<==<-->is_nothrow_move_constructible<T>::value<-->
         >
auto -->copy_or_move<--(wrap_iter<T*> begin, wrap_iter<T*>, end, Data& new_memory) -> void;
        \end{lstlisting}
    \end{block}
\end{frame}

\begin{frame}[fragile]{Tag Dispatch and SFINAE}
    \begin{block}{That's tough.}
    \pause
    However\dots
    \end{block}
\end{frame}

\begin{frame}[fragile]{C++17}
    \begin{block}{We started with this pseudo code}
        \begin{lstlisting}
auto reserve(size_type n) -> void
{
    if (capacity < n)
    {
        auto new_memory = allocate_new_memory(n);

        if (-->T is trivially copyable<--)
            do_memcpy(new_memory);

        else if (-->T is noexcept movable<--)
            move_every_item(new_memory);

        else
            copy_every_item(new_memory);
    }
}
        \end{lstlisting}
    \end{block}
\end{frame}

\begin{frame}[fragile]{C++17}
    \begin{block}{if constexpr makes the pseudo code become reality}
        \begin{lstlisting}
auto reserve(size_type n) -> void
{
    if (capacity < n)
    {
        auto new_memory = allocate_new_memory(n);

        ==>if constexpr<== (-->is_trivially_copyable_v<T><--)
            do_memcpy(new_memory);

        else ==>if constexpr<== (-->is_nothrow_move_constructible_v<T><--)
            move_every_item(new_memory);

        else
            copy_every_item(new_memory);
    }
}
        \end{lstlisting}
    \end{block}
\end{frame}


\begin{frame}[fragile]{C++17}
    \begin{block}{C++17 rocks!}
        Also look at {\tt void\_t}.
   \end{block}
\end{frame}


\begin{frame}[fragile]{CRTP}
        \pause
    \begin{block}{How do you make this work?}
        \begin{lstlisting}
#include <memory>

struct Foo;

void do_something(==>std::shared_ptr<Foo><== f);

struct Foo
{
    auto do_it()
    {
        do_something(==>???<==);
    }
};

int main() {
    auto foo = std::make_shared<Foo>();
    foo->do_it();
}
        \end{lstlisting}
    \end{block}
\end{frame}

\begin{frame}[fragile]{CRTP}
    \begin{block}{std::enable\_shared\_from\_this}
        \begin{lstlisting}
#include <memory>

struct Foo;

void do_something(std::shared_ptr<Foo> f);

struct Foo==| : public std::enable_shared_from_this<Foo>
{
    auto do_it()
    {
        do_something(-->shared_from_this()<--);
    }
};

int main() {
    auto foo = std::make_shared<Foo>();
    foo->do_it();
}
        \end{lstlisting}
    \end{block}
\end{frame}

\begin{frame}[fragile]{CRTP}
    \begin{block}{std::enable\_shared\_from\_this}
        \begin{lstlisting}
template<class _Tp>
class enable_shared_from_this
{
    --|mutable weak_ptr<_Tp> __weak_this_;
        \end{lstlisting}
\pause
        \begin{lstlisting}
protected:
    constexpr enable_shared_from_this() noexcept {}
    enable_shared_from_this(enable_shared_from_this const&) noexcept {}
    enable_shared_from_this& operator=(enable_shared_from_this const&) noexcept{ return *this; };
        \end{lstlisting}
\pause
        \begin{lstlisting}
public:
    shared_ptr<_Tp> shared_from_this()
        {return shared_ptr<_Tp>(__weak_this_);}

    shared_ptr<_Tp const> shared_from_this() const
        {return shared_ptr<const _Tp>(__weak_this_);}
        \end{lstlisting}
\pause
        \begin{lstlisting}
    --|template <class _Up> friend class shared_ptr;
};
        \end{lstlisting}
    \end{block}
\end{frame}

\begin{frame}[fragile]{sqlpp::statement\_t}
    \huge Let mix in variadics\dots
\end{frame}

\begin{frame}[fragile]{CRTP + Variadic Templates}
    \begin{block}{sqlpp11: SQL expressions in C++}
        \begin{lstlisting}
for (const auto& row : db.run(
                        select(foo.name, foo.hasFun)
                        .from(foo)
                        .where(foo.id > 17 and foo.name.like("%bar%"))))
{
  if (row.name.is_null())
    std::cerr << "name will convert to empty string" << std::endl;
  std::string name = row.name;

  bool hasFun = row.hasFun;

}
        \end{lstlisting}
    \end{block}
\end{frame}

\begin{frame}[fragile]{CRTP + Variadic Templates}
    \begin{block}{The statement}
        \begin{lstlisting}
template <typename... Clauses>
struct statement_t : public Clauses::template _base_t<detail::statement_t<Clauses...>>...,
                     //...
{
    //...
};

        \end{lstlisting}
    \end{block}
\end{frame}

\begin{frame}[fragile]{CRTP + Variadic Templates}
    \begin{block}{The statement}
        \begin{lstlisting}
struct no_from_t
{
  template <typename Statement>
  struct _base_t
  {
    template <typename Table>
    auto from(Table table) const -> _new_statement_t<check_from_t<Table>, from_t<from_table_t<Table>>>
    {
      return _from_impl(check_from_t<Table>{}, table);
    }
        \end{lstlisting}
\pause
        \begin{lstlisting}
  private:
    template <typename Check, typename Table>
    auto _from_impl(Check, Table table) const -> inconsistent<Check>;

    template <typename Table>
    auto _from_impl(consistent_t, Table table) const
        -> _new_statement_t<consistent_t, from_t<Database, from_table_t<Table>>>;
  };
};
        \end{lstlisting}
    \end{block}
\end{frame}

\begin{frame}[fragile]{Template Magic?}
    \begin{block}{No}
        \pause
        Template Perseverance
    \end{block}
\end{frame}

\begin{frame}[fragile]{}
        \huge Thank you!\\
\end{frame}

\end{document}
